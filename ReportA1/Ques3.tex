\documentclass[Space3_Assign1.tex]{subfiles}

\begin{document}
\section{Orbital Determination}

\subsection{Introduction}

\subsection{Methodology}
The perturbation model from Question 2 was used
\subsection{Results/Discussion}
http://vanallenprobes.jhuapl.edu/mission/conversation/overview/index.php\\
Three networks are used to communicate with the Van Allen Probes. The main communications ground station is at the Applied Physics Laboratory in Washington, US.\\
http://www.scf.jhuapl.edu/\\
The LLH coordinates are 39.10N 76.53W 140 AMSL\\
For backup, the satellite communicates with the United Space Network with ground stations in Hawaii and Australia. The data rates that can be achieved with this network is high enough to gather scientific data from the probe. Also as a backup, the NASA program Space Network is used for monitoring data, which relays through geosynchronous Tracking and Data Relay Satellites (TDRS).

\end{document}
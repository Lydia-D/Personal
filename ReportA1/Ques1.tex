\documentclass[Space3_Assign1.tex]{subfiles}

\begin{document}

\section{Simulation of Orbits with Classical Elements}
\subsection{Introduction}
- keplers three laws\\
- perifocal frame\\
- The true anomaly $\theta$ is the angle taken at the focus of the perifocal frame to the satellite from the perigee. The eccentric anomaly $\textit{E}$ is the angle taken at the centre of perifocal frame to the satellite from the perigee.\\
-The mean anomaly \textit{$M_t$} is the mean number of orbits per day.\\
- LEO,MEO\\
-TLE's \\



\subsection{Methodology}
From Kepler's second law, the mean anomaly at time $\textit{t}$ is calculated using the mean motion $\textit{n}$ from an epoch time described by $M_0(t_0)$. 
\begin{eqnarray}
M_t = M_0 + n(t-t_0)
\end{eqnarray}
To solve for the eccentric anomaly, newtons method was used
\begin{eqnarray}
E_{i+1} &=& E_i - \frac{f(E_i)}{f'(E_i)}\\
E_{i+1} &=& E_i - \frac{E-e\sin(E_i)-M_t}{1-e\cos(E_i)}
\end{eqnarray}


\subsection{Results/Discussion}
\subsubsection{Van Allen Probes}
The satellite RBSP-A, also known as the Van Allen Probes, is in a highly eccentric orbit. RBSP-A has a perigee in LEO at an altitude of 596 km and an apogee in MEO at an altitude of 30421 km assuming a spherical Earth. 


\subsubsection{Orbital Properties}
\begin{table}[h]
\centering
\caption{Orbital Properties - maybe put in classical parameters}
\begin{tabular}{|c|c|c|}
\hline
Orbital Properties & Van Allen Probe & Other sat \\\hline
Period & & \\\hline
Altitude at Perigee & & \\\hline
Altitude at Apogee & & \\\hline

\end{tabular}
\end{table}



\end{document}
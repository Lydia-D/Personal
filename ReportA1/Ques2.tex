\documentclass[Space3_Assign1.tex]{subfiles}

\begin{document}
\section{Simulating Perturbations}

\subsection{Introduction}
Different perturbing sources acting on Van Allen Probes - earth oblateness, gravity harmonics, solar/lunar gravity forces, aerodynamic drag, solar radiation\\

\subsubsection{Earth oblateness}
For the standard simulation model in Question 1, it was assumed that the Earth was spherical. A perfectly spherical mass has an inverse square relation of the gravitational field to the force applied on a body. However, the Earth is slightly oblate, as it is flatter at the poles and wider at the equator than a sphere. As the difference in the force is small compared to the total force, it can be mathematically treated as a perturbation. While this method is only an approximation it works well.


Newtown's second law and his law of gravitation results in 
\begin{eqnarray}
\av{r}+\frac{\mu\dv{r}}{r^3} = a_p
\end{eqnarray}

\subsection{Methodology}




\subsection{Results/Discussion}

\end{document}
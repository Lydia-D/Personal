\documentclass[Space3_Assign3.tex]{subfile}

\begin{document}
\section{Question 2}\label{Sec:Question2}

\subsection{Introduction}
The final GEO orbit is defined as having a period of one sidereal day, 23 hours 56 minutes 4.0916 seconds. Using Kepler's third law Eq\eqref{kep3} and the vis-viva law Eq\eqref{visviva} the semi-major axis and velocity of the circular orbit were calculated, see Table \ref{Table:orbitpar}.

\begin{eqnarray}
a = \sqrt[1/3]{\frac{\mu\mathbb{P}^2}{4\pi^2}} \label{kep3}\\
v = \sqrt{\frac{\mu}{a}} \label{visviva}
\end{eqnarray}

The satellite parameters in the park orbit are
\begin{table}[h]
\centering
\caption{text}
\label{Table:orbitpar}
\begin{tabularx}{0.75\linewidth}{YYY}
\toprule\toprule
\textbf{Orbital Parameter} & \textbf{Initial Value} & \textbf{Final Value} \\ \midrule
Semi-major axis & 6655937 m & \\
Period & & 86164.0916 s \\
Velocity & & \\
Eccentricity & 0 & 0 \\
Inclination angle & -28.5\Deg & 0\Deg \\
RAAN & 0\Deg & \\
Argument of Perigee & 0\Deg & \\
Mean Anomaly & 0\Deg & free \\
Epoch & 0 s & free \\\bottomrule\bottomrule 
\end{tabularx}
\end{table}

\subsection{Cost and Constraints}
What are constraints - why are they important.\\
The constraints on this optimisation problem defines the final GEO orbit. They are  The radius and velocity are enforced by requiring the ratio between the 
\begin{eqnarray}
J = |\Delta v_1 | + |\Delta v_2| \label{costfn}
\end{eqnarray}

\subsection{Local frame of satellite}
The local frame of the satellite where the burn is applied 




\subsection{Methodology}






The Hessian of the Lagrangian $H_\lag = \nabla^2_{xx}\lag$\\
BFGS method. Approximation of the Hessian update

\begin{eqnarray}
\bm{H}_{k+1} &=& \bm{H}_k - \frac{\bm{H}_k s_k s_k^T \bm{H}_k}{s_k^T \bm{H}_k s_k} + \frac{\bm{y}_k\bm{y}_k^T}{\bm{y}_k^T s_k}\\
s_k &=& x_{k+1} - x_k\\
y_k &=& \nabla f_{k+1} - \nabla f_k
\end{eqnarray}





\end{document}
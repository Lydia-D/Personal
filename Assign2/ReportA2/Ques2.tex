\documentclass[Space3_Assign2]{subfile}

\begin{document}

\section{GLONASS Orbital Determination}

The Russian GNSS is called GLONASS, a system that requires 24 satellites in three orbital planes to have full global coverage. The satellites are in nearly circular orbits at an average altitude of 19130 km, approximate period of 11 hrs and 15 min and an inclination of 65\Deg. Each orbital plane is separated by 120\Deg in the ascending node and consists of eight satellites. There is currently 28 satellites in orbit with 23 operational. For the following analysis, only the operational satellites are used.

% Vernial equinox data
%http://earthsky.org/astronomy-essentials/everything-you-need-to-know-vernal-or-spring-equinox
%http://science.nasa.gov/iSat/?group=glo-ops&satellite=40001
 % data all TLE https://www.celestrak.com/NORAD/elements/glo-ops.txt but used tle from SpaceTrack

%On high latitudes (north or south), GLONASS' accuracy is better than that of GPS due to the orbital position of the satellites
\end{document}